%% ------------------------------------------------------------------------- %%
\chapter{Introdução}
\label{cap:introducao}

\section{Escopo}

O presente trabalho visa apresentar as decisões e o conjunto de propostas que irão orientar a estratégia de implantação de um programa de Engenharia de Software em uma empresa de desenvolvimento de software fictícia. Esta empresa será responsável pelo desenvolvimento de um software que atenda ao escopo apresentado no estudo de caso \textit{"Restaurante Da Gino"}.

\subsection{Premissas assumidas}

\emph{"O Gino quer abrir um restaurante: o Da Gino, com grande capacidade de lugares, pretendendo informatizar alguns aspectos do seu negócio. Sua experiência no ramo mostrou que é preciso padronizar as receitas das refeições, ter um bom controle do consumo dos itens utilizados na preparação das refeições e ter agilidade no controle da disponibilidade das mesas. A informatização deverá ajudar o cumprimento desses objetivos de negócio, pois o Gino pretende operar com uma equipe pequena de funcionários."}\\
\textbf{Local}: Novo\\
\textbf{Funcionários}: 20 funcionários incluindo o Gino

%% ------------------------------------------------------------------------- %%
\section{Organização do documento}
\label{sec:organizacao_trabalho}

No capítulo~\ref{cap:processoengenharia}, apresentamos a proposta de programa de engenharia de software implantado na empresa responsável pelo produto de software definido na disciplina. O capítulo~\ref{cap:processoengenharia} cobre as seguintes áreas de conhecimento:

\begin{enumerate}
	\item Ciclo de vida e modelo de processo~\ref{sec:modelodeprocesso}
	\item Requisitos de software~\ref{sec:requisitos}
	\item Qualidade de software~\ref{sec:qualisoftware}
	\item Gerência da engenharia de software~\ref{sec:gerenciaengenharia}
	\item Gerência da configuração de software~\ref{sec:gerenciaconfig}
	\item Teste de software~\ref{sec:teste}
\end{enumerate}

 E no capítulo~\ref{cap:conclusoes} apresentamos as considerações finais sobre o trabalho e a disciplina.


