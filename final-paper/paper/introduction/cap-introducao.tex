%% ------------------------------------------------------------------------- %%
\chapter{Introdução}
\label{cap:introducao}

\section{Objetivo}

O presente trabalho visa aplicar os princípios da disciplina de Engenharia de Software baseado no quadro de referência ISO ISO/IEC 12207:1995 e 2008, através de um estudo de caso de fornecimento e aquisição de Software para um restaurante. Usando o modelo de documento de Visão para definir o produto de software inicial e o framework de processo Ágil-SCRUM para construir um protótipo inicial do produto de software, além de definir as estimativas do projeto de software, aplicando os artefatos desta metodologia ágil. 

\section{Premissas assumidas}

\emph{"O Gino quer abrir um restaurante: o Da Gino, com grande capacidade de lugares, pretendendo informatizar alguns aspectos do seu negócio. Sua experiência no ramo mostrou que é preciso padronizar as receitas das refeições, ter um bom controle do consumo dos itens utilizados na preparação das refeições e ter agilidade no controle da disponibilidade das mesas. A informatização deverá ajudar o cumprimento desses objetivos de negócio, pois o Gino pretende operar com uma equipe pequena de funcionários."}\\
\textbf{Local}: Novo\\
\textbf{Funcionários}: 10 funcionários incluindo o Gino

%% ------------------------------------------------------------------------- %%
\section{Organização do Trabalho}
\label{sec:organizacao_trabalho}

No capítulo~\ref{cap:modeloprocesso}, apresentamos o modelo de processo adotado segundo o quadro de referência ISO/IEC 12207:1995 e 2008. Já o capítulo~\ref{cap:requisitos} apresentamos os requisitos e a o projeto Design do software a ser implantando. No capítulo~\ref{cap:testeconfiguracao} apresentamos a construção do software e a gerência de configuração. E no capítulo~\ref{cap:qualigerencia} discutimos qualidade de software e gerência da engenharia de software.


