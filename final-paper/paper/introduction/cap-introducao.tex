%% ------------------------------------------------------------------------- %%
\chapter{Introdução}
\label{cap:introducao}

\section{Objetivo}

O presente trabalho visa apresentar as decisões e o conjunto de propostas que irão orientar a estratégia de implantação de um programa de Engenharia de Software em uma empresa de desenvolvimento de software fictícia. Esta empresa será responsável pelo desenvolvimento de um software que atenda ao escopo apresentado no estudo de caso \textit{"Restaurante Da Gino"}.

\section{Premissas assumidas}

\emph{"O Gino quer abrir um restaurante: o Da Gino, com grande capacidade de lugares, pretendendo informatizar alguns aspectos do seu negócio. Sua experiência no ramo mostrou que é preciso padronizar as receitas das refeições, ter um bom controle do consumo dos itens utilizados na preparação das refeições e ter agilidade no controle da disponibilidade das mesas. A informatização deverá ajudar o cumprimento desses objetivos de negócio, pois o Gino pretende operar com uma equipe pequena de funcionários."}\\
\textbf{Local}: Novo\\
\textbf{Funcionários}: 20 funcionários incluindo o Gino

%% ------------------------------------------------------------------------- %%
\section{Organização do Trabalho}
\label{sec:organizacao_trabalho}

No capítulo~\ref{cap:processoengenharia}, apresentamos o modelo de processo adotado segundo o quadro de referência ISO/IEC 12207:1995 e 2008 \cite{iso12207:95, iso12207:2008}. Já o capítulo~\ref{cap:requisitos} apresentamos os requisitos e o projeto Design do software a ser implantando. No capítulo~\ref{cap:testeconfiguracao} apresentamos a construção do software e a gerência de configuração. E no capítulo~\ref{cap:qualigerencia} discutimos qualidade de software e gerência da engenharia de software.


