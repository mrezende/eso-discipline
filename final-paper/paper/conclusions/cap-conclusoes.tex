%% ------------------------------------------------------------------------- %%
\chapter{Conclusões}
\label{cap:conclusoes}

%------------------------------------------------------
\section{Considerações Finais} 

\begin{description}
\item [Marcelo] Este trabalho e o curso foram importantes para mim, pois ajudou a conhecer um pouco sobre as ISOs e normas que definem o processo de engenharia de software. O significado da palavra norma no dicionário é regra, princípio e lei. Porém, ao longo do curso foi nos mostrado uma outra visão das ISOs e normas, que elas podem ser flexíveis, adaptáveis inclusive para o contexto ágil. Foi importante visualizarmos o quadro de referência e entender que o desenvolvimento de um produto de software inicia no processo de aquisição e fornecimento. E por mais que o XP e o Scrum definam práticas e técnicas para auxiliar o desenvolvimento e gerenciamento de um produto, ficou claro ao longo do curso que há muitas lacunas a serem preenchidas. Por exemplo, não há uma definição tanto em XP como no Scrum sobre como lidar com a gerência de configuração. Neste caso, o documento IEEE Std 828-2012 serve como referência e guia. O mais interessante para mim, foi mudar a minha visão e opinião em relaçãos as ISOs e normas, que antes eu via como algo burocrático. Para mim, ficou claro que elas servem como guia e referência. Elas definem as atividades e tarefas que devem (ou deveriam) ser feitas, porém não mostram como devem ser feitas. Pois a maneira como deve ser feita vai depender do contexto do projeto, há diversas variáveis envolvidas, não existe uma solução única. E as normas, ISOs e, principalmente, o SWEBOK tem uma linguagem direta e clara, isto facilita bastante para quem deseja utilizá-la. Tanto o trabalho, apresentação e aulas foram essencias para ter uma visão um pouco mais clara do processo de engenharia de software.

\item [Alex] Concluímos que a importância de desenvolver um produto, que pode ser este de software ou não, começa primeiramente na definição de alguns processos fundamentais: Aquisição e Fornecimento. Processos estes que são bases e fornecem boas diretrizes que irão garantir uma melhor qualidade e uma melhor diretriz de como obter o sucesso do produto a ser desenvolvido. As estórias de usuário e testes de aceitação oferece ótimas técnicas que permite abstrair claramente as necessidades e as partes envolvidas para o desenvolvimento do produto, além de pode ser uma forte referência para o método ágil – Scrum. O método ágil destaca-se por sua agilidade e flexibilidade às mudanças, porém vale ressaltar que seu fluxo exige certa maturidade das partes envolvidas para o desenvolvimento e estimativas, e também de forte interação com o responsável do produto, o que pode ser um ponto positivo ou negativo dependendo da disponibilidade do mesmo.
\end{description}

