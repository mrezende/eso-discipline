%% ------------------------------------------------------------------------- %%
\chapter{Qualidade e gerência de engenharia de software}
\label{cap:qualigerencia}

Texto texto texto texto texto texto texto texto texto texto texto texto texto
texto texto texto texto texto texto texto texto texto texto texto texto texto
texto texto texto texto texto texto texto texto texto texto texto texto texto
texto texto texto texto texto texto texto texto texto texto texto texto texto
texto texto texto texto texto texto.

%% ------------------------------------------------------------------------- %%
\section{Fundamentos}\index{?ea do trabalho!fundamentos}
\label{sec:fundamentos}

Texto texto texto texto texto texto texto texto texto texto texto texto texto
texto texto texto texto texto texto texto texto texto texto texto texto texto
texto texto texto texto texto texto texto texto texto texto texto texto texto
texto texto texto texto texto texto texto texto texto texto texto texto texto
texto texto texto.

%% ------------------------------------------------------------------------- %%
\subsection{?cidos Nucl?cos}\index{?ido!nucl?co}\index{nucleot?eos}
\label{sec:acidos_nucleicos}

Na Figura~\ref{fig:humanbeta} texto texto texto texto texto texto texto texto
texto texto texto texto texto texto texto texto texto texto texto texto texto
texto texto texto texto texto texto texto texto texto texto texto texto texto
texto texto texto texto texto texto texto texto texto texto texto texto texto
texto texto texto.

\begin{figure}[!h]
  \centering
  \includegraphics[width=.40\textwidth]{graph} 
  \caption{Descri?o da figura mostrada.}
  \label{fig:humanbeta} 
\end{figure}

%% ------------------------------------------------------------------------- %%
\subsection{Amino?idos}\index{?ido!amino|(}
\label{sec:amino_acidos}

Veja na Tabela \ref{tab:amino_acidos}...  texto texto texto texto texto texto
texto texto texto texto texto texto texto texto texto texto texto texto texto
texto texto texto texto texto texto texto texto texto texto texto texto texto
texto texto texto texto texto texto texto texto texto texto texto texto texto
texto texto texto texto texto texto texto texto texto texto texto.

\begin{table}[!t]
\begin{center}
    \begin{tabular}{c|c|l}
	 \hline
	 C?igo & Abreviatura & Nome completo \\ \hline
     \texttt{A} & Ala & Alanina \\
     \texttt{C} & Cys & Ciste?a \\
     ...        & ... & ... \\
     \texttt{W} & Trp & Tiptofano \\
     \texttt{Y} & Tyr & Tirosina \\ \hline
    \end{tabular}
  \caption{C?igos, abreviaturas e nomes dos amino?idos.}
  \label{tab:amino_acidos}
\end{center}
\end{table}
\index{?ido!amino|)}

Texto texto texto texto texto texto texto texto texto texto texto texto texto
texto texto texto texto texto texto texto texto texto texto texto texto texto
texto texto texto texto texto texto texto texto texto texto texto texto texto
texto texto texto texto texto texto texto texto texto texto texto texto texto
texto texto texto texto texto texto texto.


%% ------------------------------------------------------------------------- %%
\section{Exemplo de C?igo-Fonte em Java}
\label{sec:exemplo_codigo_fonte}
Texto texto texto texto texto texto texto texto texto texto texto texto texto
texto texto texto texto texto texto texto texto texto texto texto texto texto
texto texto texto texto texto texto texto texto texto texto texto texto texto
texto texto texto texto texto texto texto.

% Foi utilizado o pacote listing para formatar c?igo fonte
% http://ctan.org/tex-archive/macros/latex/contrib/listings/listings.pdf
% Veja no preambulo do arquivo tese-exemplo.tex os par?etros de configura?o.

\begin{lstlisting}[frame=trbl]
    for(i = 0; i < 20; i++)
    {
        // Coment?io 
        System.out.println("Mensagem...");
    }
\end{lstlisting}


%% ------------------------------------------------------------------------- %%
\section{Algumas Refer?cias}
\label{sec:algumas_referencias}

?muito recomend?el a utiliza?o de arquivos \emph{bibtex} para o gerenciamento
de refer?cias a trabalhos. Nesse sentido existem tr? plataformas gratuitas
que permitem a busca de refer?cias acad?icas em formato bib: 
\begin{itemize}
	\item \emph{CiteULike} (patrocinados por Springer): \url{www.citeulike.org}
	\item Cole?o de bibliografia em Ci?cia da Computa?o: \url{liinwww.ira.uka.de/bibliography}
	\item Google acad?ico (habilitar bibtex nas prefer?cias): \url{scholar.google.com.br}
\end{itemize}
Lamentavelmente, ainda n? existe um mecanismo de verifica?o ou valida?o das
informa?es nessas plataformas. Portanto, ?fortemente sugerido validar todas
as informa?es de tal forma que as entradas bib estejam corretas.  Tamb?, tome
muito cuidado na padroniza?o das refer?cias bibliogr?icas: ou considere TODOS
os nomes dos autores por extenso, ou TODOS os nomes dos autores abreviados.
Evite misturas inapropriadas.

Exemplos de refer?cias com a tag:
\begin{itemize}
\item @Book: \cite{JW82}.
{\scriptsize\begin{verbatim}
@Book{JW82,
 author   = {Richard A. Johnson and Dean W. Wichern},
 title    = {Applied Multivariate Statistical Analysis},
 publisher= {Prentice-Hall},
 year     = {1983}
}
\end{verbatim}}

\item @Article: \cite{MenaChalco08}.
{\scriptsize\begin{verbatim}
@Article{MenaChalco08,
 author   = {Jes?s P. Mena-Chalco and Helaine Carrer and Yossi Zana and 
            Roberto M. Cesar-Jr.},
 title    = {Identification of protein coding regions using the modified 
            {G}abor-wavelet transform},
 journal  = {IEEE/ACM Transactions on Computational Biology and Bioinformatics},
 volume   = {5},
 pages    = {198-207},
 year     = {2008},
}
\end{verbatim}}

\item @InProceedings: \cite{alves03:simi}.
{\scriptsize\begin{verbatim}
@InProceedings{alves03:simi,
 author   = {Carlos E. R. Alves and Edson N. C?eres and Frank Dehne and 
            Siang W. Song},
 title    = {A Parallel Wavefront Algorithm for Efficient Biological 
            Sequence Comparison},
 booktitle= {ICCSA '03: The 2003 International Conference on Computational Science
            and its Applications},
 year     = {2003},
 pages    = {249-258},
 month    = May,
 publisher= {Springer-Verlag}
}
\end{verbatim}}

\item @InCollection: \cite{bobaoglu93:concepts}.
{\scriptsize\begin{verbatim}
@InCollection{bobaoglu93:concepts,
 author   = {Ozalp Babaoglu and Keith Marzullo},
 title    = {Consistent Global States of Distributed Systems: Fundamental Concepts
            and Mechanisms},
 editor   = {Sape Mullender},
 booktitle= {Distributed Systems},
 edition  = {segunda},
 year     = {1993},
 pages    = {55-96}
}
\end{verbatim}}

\item @Conference: \cite{bronevetsky02}.
{\scriptsize\begin{verbatim}
@Conference{bronevetsky02,
 author   = {Greg Bronevetsky and Daniel Marques and Keshav Pingali and 
            Paul Stodghill},
 title    = {Automated application-level checkpointing of {MPI} programs},
 booktitle= {PPoPP '03: Proceedings of the 9th ACM SIGPLAN Symposium on Principles
            and Practice of Parallel Programming},
 year     = {2003},
 pages    = {84-89}
}
\end{verbatim}}

\item @PhdThesis: \cite{garcia01:PhD}.
{\scriptsize\begin{verbatim}
@PhdThesis{garcia01:PhD,
 author   = {Islene C. Garcia},
 title    = {Vis?es Progressivas de Computa?es Distribu?as},
 school   = {Instituto de Computa?o, Universidade de Campinas, Brasil},
 year     = {2001},
 month    = {Dezembro}
}
\end{verbatim}}

\item @MastersThesis: \cite{schmidt03:MSc}.
{\scriptsize\begin{verbatim}
@MastersThesis{schmidt03:MSc,
 author   = {Rodrigo M. Schmidt},
 title    = {Coleta de Lixo para Protocolos de \emph{Checkpointing}},
 school   = {Instituto de Computa?o, Universidade de Campinas, Brasil},
 year     = {2003},
 month    = Oct
}
\end{verbatim}}

\item @Techreport: \cite{alvisi99:analysisCIC}.
{\scriptsize\begin{verbatim}
@Techreport{alvisi99:analysisCIC,
 author   = {Lorenzo Alvisi and Elmootazbellah Elnozahy and Sriram S. Rao and
            Syed A. Husain and Asanka Del Mel},
 title    = {An Analysis of Comunication-Induced Checkpointing},
 institution= {Department of Computer Science, University of Texas at Austin},
 year     = {1999},
 number   = {TR-99-01},
 address  = {Austin, {USA}}
}
\end{verbatim}}

\item @Manual: \cite{CORBA:spec}.
{\scriptsize\begin{verbatim}
@Manual{CORBA:spec,
 title    = {{CORBA v3.0 Specification}},
 author   = {{Object Management Group}},
 month    = Jul,
 year     = {2002},
 note     = {{OMG Document 02-06-33}}
}
\end{verbatim}}

\item @Misc: \cite{gridftp}.
{\scriptsize\begin{verbatim}
@Misc{gridftp,
 author   = {William Allcock},
 title    = {{GridFTP} protocol specification. {Global Grid Forum}
            Recommendation ({GFD}.20)},
 year     = {2003}
}
\end{verbatim}}

\item @Misc: para refer?cia a artigo online \cite{fowler04:designDead}.
{\scriptsize\begin{verbatim}
@Misc{fowler04:designDead,
 author   = {Martin Fowler},
 title    = {Is Design Dead?},
 year     = {2004},
 month    = May,
 note     = {?timo acesso em 30/1/2010},
 howpublished= {\url{http://martinfowler.com/articles/designDead.html}},
}
\end{verbatim}}

\item @Misc: para refer?cia a p?ina web \cite{FSF:GNU-GPL}.
{\scriptsize\begin{verbatim}
@Misc{FSF:GNU-GPL,
 author   = {Free Software Foundation},
 title    = {GNU general public license},
 note     = {?timo acesso em 30/1/2010},
 howpublished= {\url{http://www.gnu.org/copyleft/gpl.html}},
}
\end{verbatim}}

\end{itemize}

