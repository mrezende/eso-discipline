% Arquivo LaTeX de exemplo de disserta?o/tese a ser apresentados à CPG do IME-USP
%
% Vers? 5: Sex Mar  9 18:05:40 BRT 2012
%
% Cria?o: Jesus P. Mena-Chalco
% Revisão: Fabio Kon e Paulo Feofiloff
%
% Obs: Leia previamente o texto do arquivo README.txt

\documentclass[11pt,twoside,a4paper]{book}

% ---------------------------------------------------------------------------- %
% Pacotes
\usepackage{fontspec}
\usepackage{polyglossia}
\setdefaultlanguage{brazil}

\usepackage{graphicx}           % usamos arquivos pdf/png como figuras
\usepackage{setspace}                   % espaçamento flexível
\usepackage{indentfirst}                % indenta?o do primeiro par?rafo
\usepackage{makeidx}                    % ?dice remissivo
\usepackage[nottoc]{tocbibind}          % acrescentamos a bibliografia/indice/conteudo no Table of Contents
\usepackage{courier}                    % usa o Adobe Courier no lugar de Computer Modern Typewriter
\usepackage{type1cm}                    % fontes realmente escal?eis
\usepackage{listings}                   % para formatar código-fonte (ex. em Java)
\usepackage{titletoc}
%\usepackage[bf,small,compact]{titlesec} % cabe?lhos dos t?ulos: menores e compactos

\usepackage[font=small,format=plain,labelfont=bf,up,textfont=it,up]{caption}
\usepackage[usenames,svgnames,dvipsnames]{xcolor}
\usepackage[a4paper,top=2.54cm,bottom=2.0cm,left=2.0cm,right=2.54cm]{geometry} % margens
%\usepackage[pdftex,plainpages=false,pdfpagelabels,pagebackref,colorlinks=true,citecolor=black,linkcolor=black,urlcolor=black,filecolor=black,bookmarksopen=true]{hyperref} % links em preto
\usepackage[plainpages=false,pdfpagelabels,pagebackref,colorlinks=true,citecolor=DarkGreen,linkcolor=NavyBlue,urlcolor=DarkRed,filecolor=green,bookmarksopen=true]{hyperref} % links coloridos
\usepackage[all]{hypcap}                % soluciona o problema com o hyperref e capitulos
\usepackage[square,sort,nonamebreak,comma]{natbib}  % cita?o bibliogr?ica alpha (alpha-ime.bst)
\fontsize{60}{62}\usefont{OT1}{cmr}{m}{n}{\selectfont}

% ---------------------------------------------------------------------------- %
% Cabe?lhos similares ao TAOCP de Donald E. Knuth
\usepackage{fancyhdr}
\pagestyle{fancy}
\fancyhf{}
\renewcommand{\chaptermark}[1]{\markboth{\MakeUppercase{#1}}{}}
\renewcommand{\sectionmark}[1]{\markright{\MakeUppercase{#1}}{}}
\renewcommand{\headrulewidth}{0pt}

% ---------------------------------------------------------------------------- %
\graphicspath{{./figuras/}}             % caminho das figuras (recomend?el)
\frenchspacing                          % arruma o espa?: id est (i.e.) e exempli gratia (e.g.)
\urlstyle{same}                         % URL com o mesmo estilo do texto e n? mono-spaced
\makeindex                              % para o ?dice remissivo
\raggedbottom                           % para n? permitir espa?s extra no texto
\fontsize{60}{62}\usefont{OT1}{cmr}{m}{n}{\selectfont}
\cleardoublepage
\normalsize

% ---------------------------------------------------------------------------- %
% Op?es de listing usados para o c?igo fonte
% Ref: http://en.wikibooks.org/wiki/LaTeX/Packages/Listings
\lstset{ %
language=Java,                  % choose the language of the code
basicstyle=\footnotesize,       % the size of the fonts that are used for the code
numbers=left,                   % where to put the line-numbers
numberstyle=\footnotesize,      % the size of the fonts that are used for the line-numbers
stepnumber=1,                   % the step between two line-numbers. If it's 1 each line will be numbered
numbersep=5pt,                  % how far the line-numbers are from the code
showspaces=false,               % show spaces adding particular underscores
showstringspaces=false,         % underline spaces within strings
showtabs=false,                 % show tabs within strings adding particular underscores
frame=single,	                % adds a frame around the code
framerule=0.6pt,
tabsize=2,	                    % sets default tabsize to 2 spaces
captionpos=b,                   % sets the caption-position to bottom
breaklines=true,                % sets automatic line breaking
breakatwhitespace=false,        % sets if automatic breaks should only happen at whitespace
escapeinside={\%*}{*)},         % if you want to add a comment within your code
backgroundcolor=\color[rgb]{1.0,1.0,1.0}, % choose the background color.
rulecolor=\color[rgb]{0.8,0.8,0.8},
extendedchars=true,
xleftmargin=10pt,
xrightmargin=10pt,
framexleftmargin=10pt,
framexrightmargin=10pt
}

% ---------------------------------------------------------------------------- %
% Corpo do texto
\begin{document}
\frontmatter
% cabe?lho para as p?inas das se?es anteriores ao cap?ulo 1 (frontmatter)
\fancyhead[RO]{{\footnotesize\rightmark}\hspace{2em}\thepage}
\setcounter{tocdepth}{2}
\fancyhead[LE]{\thepage\hspace{2em}\footnotesize{\leftmark}}
\fancyhead[RE,LO]{}
\fancyhead[RO]{{\footnotesize\rightmark}\hspace{2em}\thepage}

\onehalfspacing  % espa?mento

% ---------------------------------------------------------------------------- %
% CAPA
% Nota: O t?ulo para as disserta?es/teses do IME-USP devem caber em um
% orif?io de 10,7cm de largura x 6,0cm de altura que h?na capa fornecida pela SPG.
\thispagestyle{empty}
\begin{center}
    \vspace*{2.3cm}
    \textbf{\Large{Implantação de um programa de Engenharia de Software em uma empresa de desenvolvimento de software}}\\
    \vspace*{1.2cm}

    \Large{Engenharia de Software - 2017 - 3º Quadrimestre}\\
	\Large{Prof. Dr. Paulo Sérgio Muniz Silva}

    \vspace*{1.2cm}
    \Large{Alex Tito de Morais}\\
    \Large{Marcelo de Rezende Martins}

    \vskip 0.5cm
    \normalsize{\today}
\end{center}






% ---------------------------------------------------------------------------- %
% Resumo
\chapter*{Resumo}

\noindent  \textbf{Estudo de caso: Restaurante \textit{Da Gino}}.\\



Estudo de caso do Restaurante \textit{Da Gino}. Este trabalho visa apresentar as decisões e o conjunto de propostas que irão orientar a estratégia de implantação de um programa de Engenharia de Software em uma empresa de desenvolvimento de software fictícia.\\ 



\noindent \textbf{Palavras-chave:} engenharia de software, scrum, métodos ágeis, modelo de processo.



% ---------------------------------------------------------------------------- %
% Sumário
\tableofcontents    % imprime o sumário



% ---------------------------------------------------------------------------- %
% Listas de figuras e tabelas criadas automaticamente
\listoffigures
\listoftables

% ---------------------------------------------------------------------------- %
% Cap?ulos do trabalho
\mainmatter

% cabeçalho para as páinas de todos os capítulos
\fancyhead[RE,LO]{\thesection}

\singlespacing              % espaçamento simples
%\onehalfspacing            % espaçamento um e meio

%% ------------------------------------------------------------------------- %%
\chapter{Introdução}
\label{cap:introducao}

\section{Objetivo}

O presente trabalho visa aplicar os princípios da disciplina de Engenharia de Software baseado no quadro de referência ISO ISO/IEC 12207:1995 e 2008, através de um estudo de caso de fornecimento e aquisição de Software para um restaurante. Usando o modelo de documento de Visão para definir o produto de software inicial e o framework de processo Ágil-SCRUM para construir um protótipo inicial do produto de software, além de definir as estimativas do projeto de software, aplicando os artefatos desta metodologia ágil. 

\section{Premissas assumidas}

\emph{"O Gino quer abrir um restaurante: o Da Gino, com grande capacidade de lugares, pretendendo informatizar alguns aspectos do seu negócio. Sua experiência no ramo mostrou que é preciso padronizar as receitas das refeições, ter um bom controle do consumo dos itens utilizados na preparação das refeições e ter agilidade no controle da disponibilidade das mesas. A informatização deverá ajudar o cumprimento desses objetivos de negócio, pois o Gino pretende operar com uma equipe pequena de funcionários."}\\
\textbf{Local}: Novo\\
\textbf{Funcionários}: 10 funcionários incluindo o Gino

%% ------------------------------------------------------------------------- %%
\section{Organização do Trabalho}
\label{sec:organizacao_trabalho}

No capítulo~\ref{cap:modeloprocesso}, apresentamos o modelo de processo adotado segundo o quadro de referência ISO/IEC 12207:1995 e 2008. Já o capítulo~\ref{cap:requisitos} apresentamos os requisitos e a o projeto Design do software a ser implantando. No capítulo~\ref{cap:testeconfiguracao} apresentamos a construção do software e a gerência de configuração. E no capítulo~\ref{cap:qualigerencia} discutimos qualidade de software e gerência da engenharia de software.


        % associado ao arquivo: 'cap-introducao.tex'
%% ------------------------------------------------------------------------- %%
\chapter{Modelo de processo de software}
\label{cap:modeloprocesso}

%% ------------------------------------------------------------------------- %%
\section{Processo de aquisição}
\label{sec:aquisicao}

Segundo a NBR ISO/IEC 12207:1998 \cite{iso12207:95}, o processo de aquisição é composto pelas seguintes atividades:

\begin{enumerate}
  \item Iniciação
  \item Preparação de pedido de proposta
  \item Preparação e atualização do contrato
  \item Monitoração do fornecedor
  \item Aceitação e conclusão
\end{enumerate}

Segundo o item 5.1.1.1 da ISO \cite{iso12207:95}, o adquirente (Gino) inicia o processo com a
descrição de um conceito ou necessidade a adquirir. Principais necessidades do adquirente são:

\begin{itemize}
  \item Padronizar catálogo de receitas
  \item Controle de estoque
  \item Controle de disponibilidade de mesas
\end{itemize}

Segundo o item 5.1.1.4, o adquirente (Gino) pode executar a definição e a análise dos requisitos do software por conta própria ou pode manter acordo com um fornecedor para executar essa tarefa. O adquirente optou por manter um acordo com a empresa de software para que seja feita a definição e análise dos requisitos. Documento de visão será criado para detalhar melhor os requisitos e a estrutura organizacional.



%% ------------------------------------------------------------------------- %%
\section{Processo de fornecimento}
\label{sec:fornecimento}

O fornecedor será responsável por analisar todos os requisitos do sistema de acordo com as necessidades levantadas pelo adquirente (Gino), através de um acordo de contrato. A proposta de tipo de contrato terá escopo variável. Quanto à responsabilidade das organizações, o fornecedor deverá atender as necessidades estabelecidas pelo adquirente para o aceite do software desenvolvido e é de responsabilidade do adquirente prover todas as informações e dados ao fornecedor para a definição do produto final.
O fornecedor será responsável por realizar toda a preparação necessária para elaboração do pedido de proposta do cliente, neste caso, o Gino.

Para a elaboração do pedido de proposta, o fornecedor tem a responsabilidade pelos seguintes itens \cite{iso12207:95}:
\begin{itemize}
  \item Requisitos do sistema
  \item Declaração do escopo
  \item Lista de produtos de software
  \item Termos e condições
  \item Restrições técnicas
\end{itemize}

Após prover os itens acima, eles só serão validados mediante aprovação do cliente.

O contrato será confeccionado mediante direitos de uso, de propriedade, de autoria, de garantia e de licença \cite{iso12207:95}. O adquirente terá prioridade no suporte da aplicação por período pré-estabelecido entre as partes. Também fica pré-definido que qualquer alteração ou aditivo que ocorra no contrato, o fornecedor e o adquirente devem estar em comum acordo. Um documento aprovado por ambas as partes deve ser elaborado suportando estas modificações: análise de impacto quanto a prazos, cronograma e custos \cite{iso12207:95}.

A aceitação será realizada de acordo com o descrito em cada item de requisito levantado, durante as entregas parciais. Uma vez que o fornecedor faz uso de um modelo de desenvolvimento que prevê entregas parciais, o adquirente poderá fazer a validação, verificação e aceitação destas entregas evoluindo até a aceitação final do projeto. A monitoração será realizada de acordo com os status das entregas parciais providas pelo fornecedor. 

Segue a descrição das tarefas e atividades do fornecedor para o processo de fornecimento:


\subsection{Iniciação}

Fornecedor conduzirá uma revisão das necessidades levantadas pelo adquirente para decidir ou propor mudanças para a aceitação do contrato (5.2.1.1 e 5.2.1.2) \cite{iso12207:95}.

\subsection{Preparação de resposta}

O fornecedor será responsável por definir todos os requisitos em resposta ao pedido do \textit{Gino}.

\subsection{Contrato}

Contrato terá escopo variável para atender as necessidades do adquirente de acordo com suas prioridades e trabalhar de acordo com o modelo de desenvolvimento da empresa, SCRUM.

\subsection{Planejamento}

\begin{itemize}
  \item Recursos internos para o desenvolvimento do software utilizando o modelo iterativo - SCRUM (5.2.4.4)
  \item Requisitos e prioridades serão descritos pelo documento de Visão (5.2.4.1)
  \item Estrutura organizacional e cada ciclo será definido pelo SCRUM (5.2.4.2 e 5.2.4.5 a.)
  \item Uso do Kanban e diagrama de Burndown para realizar acompanhamento do progresso (5.2.4.5 n.)
  \item Estes diagramas de \textit{Burndown} poderão ser disponibilizados periodicamente ao adquirente para acompanhamento do progresso
  \item Será provido treinamento ao adquirente para a utilização do produto do software (5.2.4.5 o.)
\end{itemize}

\subsection{Execução e controle}

Monitoramento de progresso feito por Kanban e Burndown (5.2.5.3).

\subsection{Revisão e avaliação}

Fornecedor fará uso de um modelo de entregas parciais. O adquirente irá revisar de acordo com a descrição em cada item de requisito levantando durante as entregas parciais.

\subsection{Entrega e conclusão}

O adquirente poderá fazer a validação e aceitação das entregas parciais evoluindo até a aceitação do projeto final.

%% ------------------------------------------------------------------------- %%
\section{Processo de desenvolvimento}
\label{sec:desenvolvimento}

Devido ao conhecimento e experiência do fornecedor (empresa de desenvolvimento de software), foi definido o modelo iterativo SCRUM com metodologias ágeis para o desenvolvimento do software. 

Segue uma descrição das atividades e tarefas do processo de desenvolvimento:

\subsection{Implementação do processo}

Para o desenvolvimento do software para o adquirente, o modelo de ciclo de vida de software definido é o modelo iterativo SCRUM utilizando metodologias ágeis. Para a correta implementação do processo, os desenvolvedores e gerentes deverão ler o livro do SCRUM.

\begin{figure}[!h]
  \centering
  \includegraphics[width=1\textwidth]{lifecyclemodel/images/agileLifecycleDetailed} 
  \caption{Seleção do ciclo de vida ágil baseado no Scrum \cite{ambysoft:09}}
  \label{fig:scrumlifecycle} 
\end{figure}

\subsection{Análise dos requisitos do sistema}

\begin{itemize}
  \item Os requisitos serão definidos pelo fornecedor de acordo com as necessidades levantadas pelo adquirente

  \item Todos os requisitos serão definidos e listados no Product Backlog

  \item Eles serão listados e priorizados pelo PO (Product Owner) e Scrum Master antes do Sprint Planning
\end{itemize}

\subsection{Projeto de arquitetura do sistema}

Mais simples possível, segundo o manifesto ágil \cite{beck2001agile, BecAnd04extreme}. Simplicidade é:
\emph{"A arte de maximizar a quantidade de trabalho não feito."}


\subsection{Análise dos requisitos do software}

Sprint Planning.

\subsection{Projeto da arquitetura do sistema}

Manter mais simples possível. A arquitetura deve ser fácil para extender e modificá-la \cite{beck2001agile, BecAnd04extreme}.

\subsection{Projeto detalhado do software}

Mais simples possível para dar liberdade aos desenvolvedores \cite{beck2001agile, BecAnd04extreme}.

\subsection{Codificação e testes do software}

Programação pareada para funcionalidades complexas e TDD (Test driven development).

\subsection{Integração do software}

Integração contínua.

\subsection{Teste de qualificação do software}

Teste de integração ao final de cada sprint.

\subsection{Integração do sistema}

Integração contínua automatizada.

\subsection{Teste de qualificação do sistema}

Teste de acordo com os requisitos do sistema feitas pelo adquirente. Por exemplo, testar o cenário de aviso de pedidos prontos ao garçom num painel.

\subsection{Instalação do software}

Fornecedor irá prover manual de instalação.

\subsection{Apoio à aceitação do software}

O fornecedor irá prover treinamento ao adquirente.


\section{Processo de operação}
\label{sec:operacao}

Segundo a norma NBR ISO/IEC 12207:1998, o processo de operação é composto pelas seguintes atividades:

\begin{enumerate}
  \item Implementação do processo
  \item Teste operacional
  \item Operação do sistema
  \item Suporte ao usuário
\end{enumerate}

\subsection{Implementação do processo}

Para registrar, resolver e rastrear os problemas será utilizado o BugTracking. Problemas identificados serão incluídos no processo de resolução de problemas.%~\ref{sec:resolucao}.

\subsection{Teste operacional}

O teste será feito pelo desenvolvedor juntamente com o adquirente para liberação do produto de software.

\subsection{Operação do sistema}

Será disponibilizado um ambiente com as mesmas condições especificadas pelo adquirente para realizar os testes operacionais.

\subsection{Suporte ao usuário}

As solicitações do usuário, quando necessário, serão encaminhados para resolução no processo de manutenção.

\section{Processo de manutenção}
\label{sec:manutencao}

Segundo a norma NBR ISO/IEC 12207:1998, o processo de manutenção é composto pelas seguintes atividades:

\begin{enumerate}
  \item Implementação do processo
  \item Análise do problema e da modificação
  \item Implementação da modificação
  \item Revisão/aceitação da manutenção
  \item Migração
  \item Descontinuação do software
\end{enumerate}

\subsection{Implementação do processo}
Para registrar, resolver e rastrear os problemas será utilizado o BugTracking. Problemas identificados serão incluídos no processo de resolução de problemas. O gerenciamento do processo de manutenção será feito com Scrum.

\subsection{Análise do problema e da modificação}

O fornecedor fará a análise do problema e da modificação pedida pelo usuário. De acordo com o problema, definirá o nível de criticidade e o prazo para modificar.

\subsection{Implementação da modificação}

Esta atividade será feita pelo desenvolvedor durante a resolução do problema.

\subsection{Revisão/aceitação da manutenção}

O fornecedor fará a revisão junto ao adquirente da modificação solicitada.

\subsection{Migração}

Não haverá migração, pois é um sistema novo.

\subsection{Descontinuação do software}

Caso haja uma descontinuação do software, todos os artefatos gerados durante o ciclo de vida do processo de software serão disponibilizados ao cliente, bem como manuais ou qualquer outro documento que o adquirente tenha requisitado em comum acordo.












         % associado ao arquivo: 'cap-conceitos.tex'
%% ------------------------------------------------------------------------- %%
\chapter{Requisitos}
\label{cap:requisitos}

Apresentação do Wanderlei, basicamente só acrescentar. 
%% ------------------------------------------------------------------------- %%
\chapter{Testes e gerência de configuração}
\label{cap:testeconfiguracao}

Apresentação do dia 07/Novembro
%% ------------------------------------------------------------------------- %%
\chapter{Qualidade e gerência de engenharia de software}
\label{cap:qualigerencia}

Apresentação do último grupo
%% ------------------------------------------------------------------------- %%
\chapter{Conclusões}
\label{cap:conclusoes}

Apresentar alguma conclusão

%------------------------------------------------------
\section{Considerações Finais} 

Considerações

        % associado ao arquivo: 'cap-conclusoes.tex'



% ---------------------------------------------------------------------------- %
% Bibliografia
\backmatter \singlespacing   % espaçamento simples
\bibliographystyle{alpha-ime}% citação bibliográfica alpha
\bibliography{bibliografia}  % associado ao arquivo: 'bibliografia.bib'

\end{document}
